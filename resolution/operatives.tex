\operative{重申}{文明环带作为人类在塔卫二的共同根基,其安全与完整不受任何单方面拓张行为的侵蚀,环带公约的权威应被全体签署方及开拓组织共同维护}

\operative{决定}{在文明环带边界与开拓区前沿设立若干守望前哨,由终末地工业协议回收部门协调部署,负责监视“天使”群落动向及侵蚀灾害征兆,并及时向公约签署方通报威胁等级}

\operative{授权}{公约理事会在紧急状态下调动各签署方的源石发动机与集成工业系统储备,用于环带边缘定居点的防护加固及受损补给线的快速修复,相关调用须遵循资源配给公平原则并留存完整账目}

\operative{呼吁}{所有在塔卫二从事开拓活动的势力,在环带外技术回收与遗迹探索行动中遵守公约框架下的技术共享义务,不得垄断关键协议技术,不得擅自激活可能诱发大规模能量乱流的古代设施}

\operative{强烈谴责}{裂地者及其他环带外帮派对开拓区前哨及补给车队实施的劫掠行为,此类行径严重破坏文明环带存续所需的资源链稳定,并直接威胁终末地工业及其他合法开拓组织人员的安全}

\operative{要求}{公约监督机构建立环带资源调度与开拓行动信息交换平台,定期发布各签署方协议技术回收进展及源石精炼产能数据,以增强互信、避免因情报封闭导致的误判与摩擦}

\operative{强调}{协议技术的回收、解析与再运用是文明在塔卫二得以延续的根本保障,鼓励各签署方向技术储备薄弱的开拓前哨派遣专业回收团队,并提供必要的逆向工程支援}

\operative*{表示感谢}{终末地工业及所有坚守在环带边缘与开拓区深处的守望者、工程师、干员与后勤人员,他们在天灾、敌意构装体与资源匮乏的三重压力下,以一百五十余年从未中断的拓进,守护着人类在这颗星球上的每一寸文明印记}
