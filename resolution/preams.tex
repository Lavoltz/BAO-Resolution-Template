\pream{注意到}{人类跨越星门抵达塔卫二已逾一百五十年,在荒野与天灾中构筑了文明环带,作为此世界的存续根基}

\pream{重申}{文明环带是全人类在塔卫二的共同根基,其安全与完整攸关环带公约全体签署方的存续,任何环带外拓张行为均应置于公约框架下审慎协调}

\pream{表示关切}{环带边缘定居点及前哨补给线持续遭受开拓区内“天使”生物群落的袭扰,此类构装体对智慧生命抱有强烈敌意,其活动频率的上升已对平民安全构成直接威胁}

\pream{强调}{源石发动机、集成工业系统等高能技术的部署必须严格遵守公约界定的协议作业区,未经环境影响评估与多方协商的单方面技术激活,可能引发不可逆的侵蚀连锁反应或能量乱流灾害}

\pream{认识到}{塔卫二地表遍布未探明的天灾路径与旧时代遗迹,环带守护力量的巡逻与探索任务需兼顾技术回收使命与人员安全,在资源匮乏的开拓前沿维持平衡尤为艰难}

\pream{回顾}{开拓者们跨越星门的决心,以及在一百五十二年前将文明的种子重新播撒于此地的“回归”之义,并坚信延续与探索是文明轨迹永恒的主题}

\pream{重申}{愿以一切必要方式,保障环带公约所确立的技术主权与共有资源不可分割原则,维护文明环带作为人类在塔卫二最后壁垒的完整与尊严}
